%
%
%        CHAPTER 2
%        SOME CHAPTER TITLE
%
%

\section{\MakeUppercase{Some Chapter Title}}\label{sec:ch2}

\subsection{Example of level 2 section \& tables}

Below are two table examples: (1) a \emph{\LaTeX} table, and {2) a \emph{TikZ} table.\footnote{Example of a footnote in Chapter~\ref{sec:ch2}.} The former is shown in Figure~\ref{tab:tbl1} below, while the latter is in Figure~\ref{tab:tbl2} on p.~\pageref{tab:tbl2}. Another example is of the landscape table, which is constructed by using \emph{TikZ} coding, see Figure~\ref{tab:tbl3} on p.~\pageref{tab:tbl3}.  

\begin{table}[!ht]
    \vspace{.4in}
    \captionsetup{width=.5\linewidth,labelfont=bf}
    \caption{Example of a \emph{\LaTeX} table}
    \vspace{.2in}
    \centering
    \begin{tabular}{lrr}
        \toprule
        Item   & Quantity & Price\textsuperscript{a}  \\
        \midrule
        Apple  & 5        & \$0.99 \\
        Orange & 3        & \$0.75 \\
        Banana & 4        & \$0.50 \\
        \bottomrule
    \end{tabular}
    \vspace{.2in}
    \caption*{Note: \textsuperscript{a}Price is provided in the \ac{usd}. \par\vspace{.15in} Source: some source}
    \label{tab:tbl1}
    \vspace{.2in}
\end{table}

\lipsum[1]

\begin{table}[!ht]
    \vspace{.4in}
    \captionsetup{width=.9\linewidth,labelfont=bf}
    \caption{Example of a \emph{TikZ} table}
    \vspace{.2in}
    \centering
    \begin{tikzpicture}
    % Individual A -----------------------------------------------------
        \matrix [matrix anchor=north, matrix of nodes, nodes in empty cells,
             nodes = {text width=2.3in, minimum height=.2in, align=left},
             column sep=1em,
             row 2/.style={align=center, font=\bfseries},
            ] (n) at (8,4) 
        {
         & \\
        Assets & Liabilities \& Net Worth \\
        Financial Assets ($FA$) & Financial Liabilities ($FL$) \\
        Non-Financial Assets ($NFA$) & Net Worth ($NW$) \\
         };
        \node[fit=(n-1-1)(n-1-2)]%
        {\textit{\MakeUppercase{Balance sheet}\textsuperscript{a}}};
        \draw[thin]  (n-2-2.south -| n.west) -- (n-2-2.south -| n.east);
        \draw[thin]  (n-1-1.south east) -- (n-4-1.south east);
    \end{tikzpicture}
    \vspace{.2in}
    \caption*{Note: \textsuperscript{a}Denominated in the \ac{usd}. \par\vspace{.15in} Source: some source}
    \label{tab:tbl2}
    \vspace{.2in}
\end{table}

\lipsum[2]

\begin{sidewaystable}[htbp]
    \vspace{.4in}
    \captionsetup{width=1\linewidth,labelfont=bf}
    \caption{Example of a landscape \emph{TikZ} table}
    \vspace{.2in}
    \centering
    \begin{tikzpicture}
    % Individual A -----------------------------------------------------
        \matrix [matrix anchor=north, matrix of nodes, nodes in empty cells,
             nodes = {text width=.8in, minimum height=.2in, align=right},
             column sep=1em,
             row 1/.style={font=\bfseries},
             column 1/.style={nodes={align=left}}
            ] (n) %at (8,4) 
        {
        Items\textsuperscript{a}     
                    & Jan   & Feb   & Mar   & Apr   & May   & Jun   & Jul   \\
        Item 1      & 1,000 & 1,000 & 1,000 & 1,000 & 1,000 & 1,000 & 1,000 \\
        Item 2      & 1,000 & 1,000 & 1,000 & 1,000 & 1,000 & 1,000 & 1,000 \\
        Item 3      & 1,000 & 1,000 & 1,000 & 1,000 & 1,000 & 1,000 & 1,000 \\
        Item 4      & 1,000 & 1,000 & 1,000 & 1,000 & 1,000 & 1,000 & 1,000 \\
        Item 5      & 1,000 & 1,000 & 1,000 & 1,000 & 1,000 & 1,000 & 1,000 \\
        Item 6      & 1,000 & 1,000 & 1,000 & 1,000 & 1,000 & 1,000 & 1,000 \\
        Item 7      & 1,000 & 1,000 & 1,000 & 1,000 & 1,000 & 1,000 & 1,000 \\
        Item 8      & 1,000 & 1,000 & 1,000 & 1,000 & 1,000 & 1,000 & 1,000 \\
        Item 9      & 1,000 & 1,000 & 1,000 & 1,000 & 1,000 & 1,000 & 1,000 \\
        Item 10     & 1,000 & 1,000 & 1,000 & 1,000 & 1,000 & 1,000 & 1,000 \\
        Item 11     & 1,000 & 1,000 & 1,000 & 1,000 & 1,000 & 1,000 & 1,000 \\
        Item 12     & 1,000 & 1,000 & 1,000 & 1,000 & 1,000 & 1,000 & 1,000 \\
        Item 13     & 1,000 & 1,000 & 1,000 & 1,000 & 1,000 & 1,000 & 1,000 \\
        Item 14     & 1,000 & 1,000 & 1,000 & 1,000 & 1,000 & 1,000 & 1,000 \\
        Item 15     & 1,000 & 1,000 & 1,000 & 1,000 & 1,000 & 1,000 & 1,000 \\
        };
        \draw[thin]  (n-1-2.south -| n.west) -- (n-1-2.south -| n.east);
        \draw[thin]  (n-1-1.north east) -- (n-16-1.south east);
        \draw[thin]  (n-16-2.south -| n.west) -- (n-16-2.south -| n.east);
    \end{tikzpicture}
    \vspace{.2in}
    \caption*{Note: \textsuperscript{a}Example of note with a table. \par\vspace{.15in} Source: some source}
    \label{tab:tbl3}
    \vspace{.2in}
\end{sidewaystable}

